% !TEX TS-program = pdflatex
% !TEX encoding = UTF-8 Unicode

% This is a simple template for a LaTeX document using the "article" class.
% See "book", "report", "letter" for other types of document.

\documentclass[11pt]{article} % use larger type; default would be 10pt

\usepackage[utf8]{inputenc} % set input encoding (not needed with XeLaTeX)

%%% Examples of Article customizations
% These packages are optional, depending whether you want the features they provide.
% See the LaTeX Companion or other references for full information.

%%% PAGE DIMENSIONS
\usepackage{geometry} % to change the page dimensions
\geometry{a4paper} % or letterpaper (US) or a5paper or....
% \geometry{margin=2in} % for example, change the margins to 2 inches all round
% \geometry{landscape} % set up the page for landscape
%   read geometry.pdf for detailed page layout information

\usepackage{graphicx} % support the \includegraphics command and options

% \usepackage[parfill]{parskip} % Activate to begin paragraphs with an empty line rather than an indent

%%% PACKAGES
\usepackage{booktabs} % for much better looking tables
\usepackage{array} % for better arrays (eg matrices) in maths
\usepackage{paralist} % very flexible & customisable lists (eg. enumerate/itemize, etc.)
\usepackage{verbatim} % adds environment for commenting out blocks of text & for better verbatim
\usepackage{subfig} % make it possible to include more than one captioned figure/table in a single float
\usepackage{amsmath}
% These packages are all incorporated in the memoir class to one degree or another...



\begin{document}
\section{calculate velocity from acceleration}
Given:\\
current Time: $ T$\\
start Time: $T_0$\\\\
Velocity: $ a(t)$\\
$a(T_0) = 0$\\\\
We know:\\
$ a = \dot{v}$ \\
 we need to calculate \\
\begin{align}
v = \int_{T_0=0}^T a(t) ~dt
\end{align}
(1)  corresponds to the area below a graph from $T_0 = 0$ to .\\
Assumtion: time intervalls are short enough to aproximate the acceleration linear\\
Time at the $i$-th timestep: $T_i$\\
Acceleration at the $i$-th timestep: $a(T_i)$\\
To calculate the velocity after $n$ timesteps we need to evaluate:
\begin{align}
 v(n) = \sum_{i=1}^{n} \frac{1}{2} (a(T_{i-1})+ a(T_i)) \cdot (T_i - T_{i-1})
\end{align}\\
Assumtion: time intervalls are short enough to aproximate the acceleration constant
\begin{align}
 v(n) = \sum_{i=1}^{n} a(T_i) \cdot (T_i - T_{i-1})
\end{align}\\\\
(2) and (3) can be calculated step by step. As we want to know the speed at any time anyway, we only need to calculate one summand at once and add it to the old velocity.

\end{document}
